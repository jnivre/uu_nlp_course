
\documentclass[11pt]{article}
%\usepackage[top=20mm,left=20mm,right=20mm,bottom=15mm,a4paper]{geometry} % see geometry.pdf on how to lay out the page. There's lots.
\usepackage[top=20mm,left=20mm,right=20mm,bottom=15mm,headsep=15pt,footskip=15pt,a4paper]{geometry} % see geometry.pdf on how to lay out the page. There's lots.
%\geometry{a4paper} % or letter or a5paper or ... etc
% \geometry{landscape} % rotated page geometry
\usepackage[round]{natbib}
\setlength{\bibsep}{0.0pt}
\usepackage{color}
\usepackage{times}
%\usepackage[T1]{fontenc}
%\usepackage{mathptmx}
\usepackage{tikz-dependency}
\usepackage{enumitem}
%\usepackage{times}

\usepackage[procnames]{listings}
\usepackage{color}
\usepackage{todonotes}

\newenvironment{titlemize}[1]{%
    \paragraph{#1}
    \begin{itemize}
        \setlength\itemsep{0pt}}
    {\end{itemize}}


% See the ``Article customise'' template for come common customisations
\newcommand{\refeq}[1]{Equation~\ref{eq:#1}}
\newcommand{\reffig}[1]{Figure~\ref{fig:#1}}
\newcommand{\reftab}[1]{Table~\ref{tab:#1}}
\newcommand{\refsec}[1]{\textsection\ref{sec:#1}}
\newcommand{\newsec}[1]{\section{#1}\noindent}
%\newcommand{\newsec}[2]{\section{#1}\label{sec:#2}\noindent}
\newcommand{\newsubsec}[2]{\subsection{#1}\label{sec:#2}\noindent}
\newcommand{\argmax}{\operatornamewithlimits{argmax}} 
\newcommand{\argmin}{\operatornamewithlimits{argmin}} 

\makeatletter         
\def\@maketitle{   % custom maketitle 
\begin{center}%
{\bfseries \@title}%
{\bfseries \@author}%
\end{center}%
\smallskip \hrule \bigskip }

% custom section 
\renewcommand{\section}{\@startsection
{section}%                   % the name
{1}%                         % the level
{0mm}%                       % the indent
{-0.8\baselineskip}%            % the before skip
{0.3\baselineskip}%          % the after skip
{\bfseries\large}}% the style

% custom subsection 
\renewcommand{\subsection}{\@startsection
{subsection}%                   % the name
{2}%                         % the level
{0mm}%                       % the indent
{-0.8\baselineskip}%            % the before skip
{0.3\baselineskip}%          % the after skip
{\bfseries\large}}% the style

\renewcommand{\paragraph}{%
  \@startsection{paragraph}{4}%
  {\z@}{1.5ex \@plus 1ex \@minus .2ex}{-1em}%
  {\normalfont\normalsize\bfseries}%
}\makeatother

%\title{{\LARGE Universal Parser (UP)}\\[-8mm]
%\includegraphics[height=8mm]{RUPA}~~~~~~~~~~~~~~~~~~~~~~~~~~~~~~~~~~~~~~~~~~~~~~~~~~~~~~~~~~~~~~~~~~~~~~~~~~~\includegraphics[height=8mm]{RUPA}}
\title{{\LARGE Natural Language Processing}\\[1.5mm]{\large Assignment 4: Semantics}}
\author{}
\date{} % delete this line to display the current date

%%% BEGIN DOCUMENT
\begin{document}

\definecolor{keywords}{RGB}{255,0,90}
\definecolor{comments}{RGB}{0,0,113}
\definecolor{red}{RGB}{160,0,0}
\definecolor{green}{RGB}{0,150,0}
\definecolor{UUlight}{RGB}{230,230,230}
\definecolor{UUmedium}{RGB}{190,190,190}
\definecolor{UUdark}{RGB}{130,130,130}
\definecolor{UUred}{RGB}{153,0,0}
 

\lstset{language=Python, 
        basicstyle=\ttfamily\small, 
        keywordstyle=\color{keywords},
        commentstyle=\color{comments},
        stringstyle=\color{red},
        showstringspaces=false,
        identifierstyle=\color{green},
        procnamekeys={def,class}}

\maketitle
%\tableofcontents
%\vspace{3mm}
\section{Introduction}
\noindent This assignment involves material from lectures 11 and
12. You should have watched the relevant videos, read the relevant
chapters in the textbook and made a serious attempt at completing the
relevant labs before you attempt this assignment.  If you feel you
have done that and still find the instructions unclear, you are
welcome to email the course teachers and/or go to office hours to ask
for help.  The assignment is split into 2 sections, one about lexical
semantics, one about semantic role labelling. Each section is worth 10
points. We expect between half a page and a page for each
section. Please do not submit more than 4 pages overall.  You answers
for each section should be self-contained.

\newsec{Lexical Semantics: Error analysis}%
In this exercise you will perform an error analysis which is based on
the results you got from training the classifiers in Lab 11.  In Lab
11, you have already analyzed the performance of a classifer. However,
in addition to that it is useful to perform an error analysis. By
providing additional arguments to {\tt wsd\_classifier()}, you can get
a confusion matrix as well as a printout of all the errors.
\begin{center}
\fbox{
\scalebox{0.55}{
\lstinputlisting{code/error.py}
}}
\end{center}
The relevant arguments are {\tt confusion\_matrix=True} and {\tt
  log=True}, but note that you must also insert the argument {\tt
  distance=3} to get the right interpretation. The errors are printed
to an external file called {\tt errors.txt}. Do an error analysis of
the \textbf{best performing classifier for each target word}, answering the
following questions:
\begin{enumerate}
\item Using the confusion matrix, identify which sense is the hardest
  one for the model to estimate. Look in {\tt errors.txt} for examples
  where that hardest word sense is the correct label.  Do you see any
  patterns or systematic errors? If so, can you think of a way to
  adapt the feature representation to improve the model?
  \textcolor{UUred}{[ca. 1/2 page of discussion]}
\item Get familiar with the concept of Precision, Recall and F-Measure
  (coursebook, Chapter 13.5.3) and calculate them based on the
  contingency table of the best performing classifier for one of the
  words. Then, discuss the (possible) advantages of Precision, Recall
  and F-measure as opposed to Accuracy when you want to improve the
  performance of a tool (in our case: a classifier)
  \textcolor{UUred}{[ca. 1/2 page]}
\end{enumerate}


\newsec{Semantic Role Labeling}%
In Lab 12, you annotated some data and discussed with your classmates.
Describe this experience and discuss inter-annotator agreement.  Give
examples of cases where agreement was good and where it was bad.  Look
up measures of inter-annotator agreement on the internet and calculate
one of them.  \todo[inline]{Can we do that? :D} \todo[inline,
color=yellow!40]{FC: giving examples is maybe not sufficient? We may
  wanna ask them to reflect on the examples, give arguments for their
  reasoning behind cases which led to disagreements?}

\clearpage
\section{Grading Criteria}
To pass the assignment, you must meet all the basic criteria on all
subparts of the assignment.  To get VG, you must in addition meet some
of the additional criteria for most of subparts.

\begin{titlemize}{Basic Criteria}
    \item Answers are given in understandable English.
    \item Answers are stated clearly and coherently.
    \item Answers are essentially correct.
\end{titlemize}
\begin{titlemize}{Additional Criteria}
    \item Answers are well motivated.
    \item Answers are well illustrated.
    \item Answers reveal extensive knowledge of the textbook
      chapter(s).\todo[inline, color=yellow!40]{FC: do you think these
        need further adjustments for this particular assignment?}
\end{titlemize}

% \begin{titlemize}{Points}
%     \item 1-4: poor on each of the basic criteria
%     \item 5-6: fair on each of the basic criteria
%     \item 7: at least fair on each of the basic criteria and fair on some of
%         the additional criteria
%     \item 8: at least good on each of the basic criteria and at least fair on
%         each of the additional criteria
%     \item 9: at least good on each of the basic criteria and at least fair and
%         most times good or excellent on each of tem the additional criteria
%     \item 10: good or excellent on each of the basic and additional criteria.
% \end{titlemize}

\section{Submit the Assignment}
Submit your assignment as a pdf file named
firstname\_lastname\_assignment\_4.pdf. It should follow the style and
margins given in the example submission even if not created with
LaTeX. The submission is due on \textit{studentportalen} before Friday %
December 15th at 20h00. Later submissions will be considered failed
submissions and assessed after the final re-submission deadline on
January 15th.\todo[inline, color=green!50]{FC: added submission format
  requirements}

% \todo[inline, color=blue!40]{FC: we should decide whether or not we
%   want them to submit in LaTeX}
% \noindent
% Submit your assignment as a pdf file named
% firstname\_lastname\_assignment\_2.pdf. It should follow the style and
% margins given in the example submission even if not created with
% LaTeX. The submission is due on \textit{studentportalen} before Friday
% 15th December 20h00. Later submissions will be considered failed
% submissions and assessed after the final re-submission deadline on
% January 15th. To pass the assignment, you must have answered both
% sections, reached at least 5 points in each section and at least 12
% overall. To get VG, you should obtain at least 8 in each question and
% at least 18 points overall.\todo[inline, color=yellow!40]{FC: This
%   sentence is a bit redundant. They cannot get 18 points without
%   getting at least 8 points in each question anyway, right? Moreover,
%   I changed the wording here from 'questions' to 'sections' in order
%   to make it fit with the text in the introduction where we say that
%   each \textbf{section} is worth 10 points.} \todo[inline,
% color=blue!40]{FC: maybe say ``must have answered \underbar{all}
%   questions?''. Sometimes, the large questions contain sub-questions.}
% \todo[inline, color=blue!40]{FC: I added the re-submission policy here
%   and will add it to the webpage as well.}

\end{document}
