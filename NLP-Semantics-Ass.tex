
\documentclass[11pt]{article}
%\usepackage[top=20mm,left=20mm,right=20mm,bottom=15mm,a4paper]{geometry} % see geometry.pdf on how to lay out the page. There's lots.
\usepackage[top=20mm,left=20mm,right=20mm,bottom=15mm,headsep=15pt,footskip=15pt,a4paper]{geometry} % see geometry.pdf on how to lay out the page. There's lots.
%\geometry{a4paper} % or letter or a5paper or ... etc
% \geometry{landscape} % rotated page geometry
\usepackage[round]{natbib}
\setlength{\bibsep}{0.0pt}
\usepackage{color}
\usepackage{times}
%\usepackage[T1]{fontenc}
%\usepackage{mathptmx}
\usepackage{tikz-dependency}
\usepackage{enumitem}
%\usepackage{times}

\usepackage[procnames]{listings}
\usepackage{color}
 
 

% See the ``Article customise'' template for come common customisations
\newcommand{\refeq}[1]{Equation~\ref{eq:#1}}
\newcommand{\reffig}[1]{Figure~\ref{fig:#1}}
\newcommand{\reftab}[1]{Table~\ref{tab:#1}}
\newcommand{\refsec}[1]{\textsection\ref{sec:#1}}
\newcommand{\newsec}[1]{\section{#1}\noindent}
%\newcommand{\newsec}[2]{\section{#1}\label{sec:#2}\noindent}
\newcommand{\newsubsec}[2]{\subsection{#1}\label{sec:#2}\noindent}
\newcommand{\argmax}{\operatornamewithlimits{argmax}} 
\newcommand{\argmin}{\operatornamewithlimits{argmin}} 

\makeatletter         
\def\@maketitle{   % custom maketitle 
\begin{center}%
{\bfseries \@title}%
{\bfseries \@author}%
\end{center}%
\smallskip \hrule \bigskip }

% custom section 
\renewcommand{\section}{\@startsection
{section}%                   % the name
{1}%                         % the level
{0mm}%                       % the indent
{-0.8\baselineskip}%            % the before skip
{0.3\baselineskip}%          % the after skip
{\bfseries\large}}% the style

% custom subsection 
\renewcommand{\subsection}{\@startsection
{subsection}%                   % the name
{2}%                         % the level
{0mm}%                       % the indent
{-0.8\baselineskip}%            % the before skip
{0.3\baselineskip}%          % the after skip
{\bfseries\large}}% the style

\renewcommand{\paragraph}{%
  \@startsection{paragraph}{4}%
  {\z@}{1.5ex \@plus 1ex \@minus .2ex}{-1em}%
  {\normalfont\normalsize\bfseries}%
}\makeatother

%\title{{\LARGE Universal Parser (UP)}\\[-8mm]
%\includegraphics[height=8mm]{RUPA}~~~~~~~~~~~~~~~~~~~~~~~~~~~~~~~~~~~~~~~~~~~~~~~~~~~~~~~~~~~~~~~~~~~~~~~~~~~\includegraphics[height=8mm]{RUPA}}
\title{{\LARGE Natural Language Processing}\\[1.5mm]{\large Assignment 4: Semantics}}
\author{}
\date{} % delete this line to display the current date

%%% BEGIN DOCUMENT
\begin{document}

\definecolor{keywords}{RGB}{255,0,90}
\definecolor{comments}{RGB}{0,0,113}
\definecolor{red}{RGB}{160,0,0}
\definecolor{green}{RGB}{0,150,0}
\definecolor{UUlight}{RGB}{230,230,230}
\definecolor{UUmedium}{RGB}{190,190,190}
\definecolor{UUdark}{RGB}{130,130,130}
\definecolor{UUred}{RGB}{153,0,0}
 

\lstset{language=Python, 
        basicstyle=\ttfamily\small, 
        keywordstyle=\color{keywords},
        commentstyle=\color{comments},
        stringstyle=\color{red},
        showstringspaces=false,
        identifierstyle=\color{green},
        procnamekeys={def,class}}

\maketitle
%\tableofcontents
%\vspace{3mm}

\newsec{Lexical Semantics: Error analysis}
In this exercise you will perform an error analysis which is based on the results you got from training the classifiers in Lab 11.
In Lab 11, you have already analyzed the performance of a classifer. However, in addition to that it is useful to
perform an error analysis. By providing additional arguments to {\tt
  wsd\_classifier()}, you can get a confusion matrix as well as a
printout of all the errors.
\begin{center}
\fbox{
\scalebox{0.55}{
\lstinputlisting{code/error.py}
}}
\end{center}
The relevant arguments are {\tt confusion\_matrix=True} and {\tt
  log=True}, but note that you must also insert the argument {\tt
  distance=3} to get the right interpretation. The errors are printed
to an external file called {\tt errors.txt}. Do an error analysis of
the \textbf{best performing classifier for each target word}, answering the
following questions:
\begin{enumerate}
\item Using the confusion matrix, identify which sense is the hardest
  one for the model to estimate.
\item  Look in {\tt errors.txt} for examples where that hardest word
  sense is the correct label.  Do you see any patterns or systematic
  errors? If so, can you think of a way to adapt the feature
  representation to improve the model? \textcolor{UUred}{[ca. 1/2 page of discussion]}
\item Get familiar with the concept of Precision, Recall and F-Measure (coursebook, Chapter 13.5.3) and calculate them based on the contingency table of the best performing classifier for one of the words. Then, discuss the (possible) advantages of Precision, Recall and F-measure as opposed to Accuracy when you want to improve the performance of a tool (in our case: a classifier) \textcolor{UUred}{[ca. 1/2 page]}
\end{enumerate}



\newsec{Formerly known as: Compositional Semantics (MDL)}%


\end{document}