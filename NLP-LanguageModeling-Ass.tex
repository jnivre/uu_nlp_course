
\documentclass[11pt]{article}
%\usepackage[top=20mm,left=20mm,right=20mm,bottom=15mm,a4paper]{geometry} % see geometry.pdf on how to lay out the page. There's lots.
\usepackage[top=20mm,left=20mm,right=20mm,bottom=15mm,headsep=15pt,footskip=15pt,a4paper]{geometry} % see geometry.pdf on how to lay out the page. There's lots.
%\geometry{a4paper} % or letter or a5paper or ... etc
% \geometry{landscape} % rotated page geometry
\usepackage[round]{natbib}
\setlength{\bibsep}{0.0pt}
\usepackage{color}
\usepackage{times}
%\usepackage[T1]{fontenc}
%\usepackage{mathptmx}
\usepackage{tikz-dependency}
\usepackage{enumitem}
%\usepackage{times}

\usepackage[procnames]{listings}
\usepackage{color}
 
 

% See the ``Article customise'' template for come common customisations
\newcommand{\refeq}[1]{Equation~\ref{eq:#1}}
\newcommand{\reffig}[1]{Figure~\ref{fig:#1}}
\newcommand{\reftab}[1]{Table~\ref{tab:#1}}
\newcommand{\refsec}[1]{\textsection\ref{sec:#1}}
\newcommand{\newsec}[2]{\section{#1}\label{sec:#2}\noindent}
\newcommand{\newsubsec}[2]{\subsection{#1}\label{sec:#2}\noindent}
\newcommand{\argmax}{\operatornamewithlimits{argmax}} 
\newcommand{\argmin}{\operatornamewithlimits{argmin}} 

\makeatletter         
\def\@maketitle{   % custom maketitle 
\begin{center}%
{\bfseries \@title}%
{\bfseries \@author}%
\end{center}%
\smallskip \hrule \bigskip }

% custom section 
\renewcommand{\section}{\@startsection
{section}%                   % the name
{1}%                         % the level
{0mm}%                       % the indent
{-0.8\baselineskip}%            % the before skip
{0.3\baselineskip}%          % the after skip
{\bfseries\large}}% the style

% custom subsection 
\renewcommand{\subsection}{\@startsection
{subsection}%                   % the name
{2}%                         % the level
{0mm}%                       % the indent
{-0.8\baselineskip}%            % the before skip
{0.3\baselineskip}%          % the after skip
{\bfseries\large}}% the style

\renewcommand{\paragraph}{%
  \@startsection{paragraph}{4}%
  {\z@}{1.5ex \@plus 1ex \@minus .2ex}{-1em}%
  {\normalfont\normalsize\bfseries}%
}\makeatother

%\title{{\LARGE Universal Parser (UP)}\\[-8mm]
%\includegraphics[height=8mm]{RUPA}~~~~~~~~~~~~~~~~~~~~~~~~~~~~~~~~~~~~~~~~~~~~~~~~~~~~~~~~~~~~~~~~~~~~~~~~~~~\includegraphics[height=8mm]{RUPA}}
\title{{\LARGE Natural Language Processing}\\[1.5mm]{\large Assignment 4: Language Modeling}}
\author{}
\date{} % delete this line to display the current date

%%% BEGIN DOCUMENT
\begin{document}

\definecolor{keywords}{RGB}{255,0,90}
\definecolor{comments}{RGB}{0,0,113}
\definecolor{red}{RGB}{160,0,0}
\definecolor{green}{RGB}{0,150,0}
 
\lstset{language=Python, 
        basicstyle=\ttfamily\small, 
        keywordstyle=\color{keywords},
        commentstyle=\color{comments},
        stringstyle=\color{red},
        showstringspaces=false,
        identifierstyle=\color{green},
        procnamekeys={def,class}}

\maketitle
%\tableofcontents
%\vspace{3mm}
\vspace{-2mm}
\newsec{Introduction}{intro}%
In this assignment, we are going to build and evaluate $n$-gram models %(for various values of $n$) 
for the task of probabilistic language modeling, or predicting the next word in a text. We will make use
of a standard toolkit called SRILM, and we will explore different orders of $n$ as well as different smoothing techniques. 

\paragraph{Acknowledgment:} Thanks to Emily Bender for letting us reuse
and modify an older assignment.

\newsec{Data and preprocessing}{data}%
To train our language models, we will use the same collection of Sherlock Holmes stories as in the previous exercises ({\tt holmes.txt}).
To evaluate the models, we will use three different texts:
\begin{enumerate}[noitemsep,topsep=0.2cm]
\item \emph{His Last Bow}, a collection of Sherlock Holmes stories by Conan Doyle (not included in {\tt holmes.txt})
\item \emph{The Lost World}, a novel by Conan Doyle
\item \emph{Stories by English Authors: London}, short stories written around the same time as the other works
\end{enumerate}
These texts are in the files {\tt his-last-bow.txt}, {\tt lost-world.txt}, and {\tt other-authors}, respectively.

\paragraph{Note on tokenization:} The SRILM toolkit requires tokenized text in a different format from the one we have been using so far. 
Instead of having one token per line and blank lines to separate sentences, we should now have one sentence per line and spaces to separate tokens.
To tokenize the texts, you should use {\tt tokenizer6.py}, which performs exactly the same tokenization as {\tt tokenizer5.py} but produces its 
output in the format required by SRILM.

\newsec{Train a language model}{train}%
To train a language model on {\tt holmes.txt}, we need to run a command like the following:
\begin{verbatim}
ngram-count -order 2 -text holmes-tok.txt -addsmooth 0.01 -lm bigram-add
\end{verbatim}
This commands tells SRILM to count bigrams ({\tt -order\,2}) in the training file ({\tt -file\,holmes-tok.txt})
and to create a language model ({\tt -lm\,bigram-add} with additive smoothing with ({\tt -addsmooth\,0.01}).
By varying parameters such as n-gram order and smoothing method, we can create different language models.

\newsec{Evaluate a language model}{eval}%
To evaluate a language model created as above on {\tt his-last-bow.txt}, we run:
\begin{verbatim}
ngram -order 2 -lm bigram-add -ppl his-last-bow-tok.txt
\end{verbatim}
This command tells SRILM to use a bigram model ({\tt -order\,2}) based on the previously stored model ({\tt -lm\,bigram-add}) 
to estimate the perplexity of the test file ({\tt -ppl\,his-last-bow-tok.txt}). The output generated by this command should be
something like:
\begin{verbatim}
file his-last-bow-tok.txt: 5123 sentences, 81395 words, 2685 OOVs
0 zeroprobs, logprob= -178072 ppl= 133.086 ppl1= 182.972
\end{verbatim}
This tells us that the test file contained 5123 sentences and 81395 words, of which 2685 were out-of-vocabulary (OOV),
meaning that they did not occur in the training set. The model did not find any zero probabilities (thanks to the smoothing),
the log probability was $-$178072 and the perplexity (ppl) was 133.086.\footnote{The second perplexity measure (ppl1)
is computed without end-of-sentence tokens and can be ignored for now.} Remember that perplexity, as well as the related
entropy measure, tells us how surprised or confused the model is when seeing the test data, so lower perplexity is better.

\newsec{Explore different test sets (10 P)}{sets}%
Our first test set is drawn from the same author and the same genre as
our training set. What happens if we apply the model to the other two
test sets? What happens if we apply the model to the training set
itself? Discuss the results of these experiments.

\newsec{Explore different language models (10 P)}{models}%
Our first language model was a bigram model with simple additive
smoothing. What happens if we change the $n$-gram order?  Explore at
least unigrams and trigrams on all three test sets with the same
smoothing method. Decide which model works best and switch the
smoothing method for this model to Kneser-Ney smoothing (a type of
backoff smoothing). Discuss the results of these experiments.

\paragraph{Note on SRILM options:} To change the $n$-gram order,
simply change {\tt -order\,2} to {\tt -order\,$n$} (for whatever $n$
you want to try).  Please note that the $n$-gram order used for
testing cannot be higher than the $n$-gram order used for training
(but it can be lower). To change smoothing method, replace {\tt
  -addsmooth\,0.01} by {\tt -kndiscount} (for Kneser-Ney
discounting). If you want to explore further options, consult the
SRILM manual at {\small {\tt
    http://www.speech.sri.com/projects/srilm/manpages/ngram-count.1.html}}.

\newsec{Submit the assignment}{submit}%
Submit the following to {\tt nlp-course@stp.lingfil.uu.se}
\textbf{before 20:00h November 19th}. In order to pass you must
provide answers to all exercises and achieve at least 15/20 points.

\begin{itemize}[noitemsep,topsep=0.2cm]
\item A table with perplexity results for all the models you tested on
  the three test sets (Sections 5+6).
\item A short discussion of what conclusions you can draw from these
  results (Sections 5+6).
\item \textbf{[VG]} In order to get the grade VG, reach at least 18
  points in the exercises above, and additionally describe the
  Kneser-Ney smoothing in a bit more detail (in your own words) and
  how this relates to the results you got in the exercises above.

\end{itemize}


\end{document}