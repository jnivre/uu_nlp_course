
\documentclass[11pt]{article}
%\usepackage[top=20mm,left=20mm,right=20mm,bottom=15mm,a4paper]{geometry} % see geometry.pdf on how to lay out the page. There's lots.
\usepackage[top=20mm,left=20mm,right=20mm,bottom=15mm,headsep=15pt,footskip=15pt,a4paper]{geometry} % see geometry.pdf on how to lay out the page. There's lots.
%\geometry{a4paper} % or letter or a5paper or ... etc
% \geometry{landscape} % rotated page geometry
\usepackage[round]{natbib}
\setlength{\bibsep}{0.0pt}
\usepackage{color}
\usepackage{times}
%\usepackage[T1]{fontenc}
%\usepackage{mathptmx}
\usepackage{tikz-dependency}
\usepackage{enumitem}
%\usepackage{times}

\usepackage[procnames]{listings}
\usepackage{color}
\usepackage{fancyvrb}
 

% See the ``Article customise'' template for come common customisations
\newcommand{\refeq}[1]{Equation~\ref{eq:#1}}
\newcommand{\reffig}[1]{Figure~\ref{fig:#1}}
\newcommand{\reftab}[1]{Table~\ref{tab:#1}}
\newcommand{\refsec}[1]{\textsection\ref{sec:#1}}
\newcommand{\newsec}[2]{\section{#1}\label{sec:#2}\noindent}
\newcommand{\newsubsec}[2]{\subsection{#1}\label{sec:#2}\noindent}
\newcommand{\argmax}{\operatornamewithlimits{argmax}} 
\newcommand{\argmin}{\operatornamewithlimits{argmin}} 

\makeatletter         
\def\@maketitle{   % custom maketitle 
\begin{center}%
{\bfseries \@title}%
{\bfseries \@author}%
\end{center}%
\smallskip \hrule \bigskip }

% custom section 
\renewcommand{\section}{\@startsection
{section}%                   % the name
{1}%                         % the level
{0mm}%                       % the indent
{-0.8\baselineskip}%            % the before skip
{0.3\baselineskip}%          % the after skip
{\bfseries\large}}% the style

% custom subsection 
\renewcommand{\subsection}{\@startsection
{subsection}%                   % the name
{2}%                         % the level
{0mm}%                       % the indent
{-0.8\baselineskip}%            % the before skip
{0.3\baselineskip}%          % the after skip
{\bfseries\large}}% the style

\renewcommand{\paragraph}{%
  \@startsection{paragraph}{4}%
  {\z@}{1.5ex \@plus 1ex \@minus .2ex}{-1em}%
  {\normalfont\normalsize\bfseries}%
}\makeatother

%\title{{\LARGE Universal Parser (UP)}\\[-8mm]
%\includegraphics[height=8mm]{RUPA}~~~~~~~~~~~~~~~~~~~~~~~~~~~~~~~~~~~~~~~~~~~~~~~~~~~~~~~~~~~~~~~~~~~~~~~~~~~\includegraphics[height=8mm]{RUPA}}
\title{{\LARGE Natural Language Processing}\\[1.5mm]{\large Lab 12: Semantic Role Labeling}}
\author{}
\date{} % delete this line to display the current date

%%% BEGIN DOCUMENT
\begin{document}

\definecolor{keywords}{RGB}{255,0,90}
\definecolor{comments}{RGB}{0,0,113}
\definecolor{red}{RGB}{160,0,0}
\definecolor{green}{RGB}{0,150,0}
 
\lstset{language=Python, 
        basicstyle=\ttfamily\small, 
        keywordstyle=\color{keywords},
        commentstyle=\color{comments},
        stringstyle=\color{red},
        showstringspaces=false,
        identifierstyle=\color{green},
        procnamekeys={def,class}}

\maketitle
%\tableofcontents
%\vspace{3mm}
\vspace{-2mm} \newsec{Introduction}{intro}%
In this lab, we are going to do PropBank style semantic
annotation and study inter-annotator agreement. We are going to focus
on one polysemous verb, the verb {\em count}, which occurs exactly 10
times in the training set of the English web treebank. All files
needed are available from {\tt /local/kurs/nlp/semantics/}.


\newsec{Study the lexicon }{lex}%
The lexical entry for the verb {\em count} can be found in the file
{\tt count.senses}. It is organized by (numbered) senses, with the
first sense being {\tt count.01}:

\begin{Verbatim}[fontsize=\footnotesize,xleftmargin=\parindent]
<roleset id="count.01" name="enumerate" vncls="-">
<roles>
  <role descr="counter" n="0" />
  <role descr="thing counted" n="1" />
</roles>

<example name="Count von Count">
<text>
    But this is a production designed and directed by Franco
    Zeffirelli and paid for by Sybil Harrington, who *trace* has no
    need *trace* to count her pennies, unlike Violetta, down to 20
    louis at the opera's end. 
</text>
        <arg n="0">*trace* -&gt; *trace* -&gt;  who -&gt;  Sybil Harrington</arg>
        <rel>count</rel>
        <arg n="1">her pennies</arg>
</example>

<note>
</note>
</roleset>
\end{Verbatim}
 This sense is nicknamed {\tt enumerate} and has
two roles: ARG0 = the counter; ARG1 = the thing(s) counted.  In the
example given, the ARG0 role is filled by a chain of coreferring NPs
(some of which are empty traces) ending in {\tt Sybil Harrington},
while the ARG1 role is filled by the NP {\tt her pennies}. A simpler
example would be:
\begin{Verbatim}[fontsize=\small,xleftmargin=\parindent]
[ARG0 Sybil Harrington] doesn't count.01 [ARG1 her pennies].
\end{Verbatim}
Study the rest of {\tt count.senses} and then answer the following
questions:
\begin{enumerate}[noitemsep,topsep=0.2cm]
\item  How many senses are there altogether?
\item  Construct one example of your own for each sense.
\end{enumerate}

\newsec{Annotate some data }{annotate}%
The file {\tt count.instances} contains all the sentences in the
training set of the English web treebank that contains the verb lemma
{\tt count}. (The file format is the CoNLL-X format that you
previously used in the dependency parsing lab.) Your task is to
annotate each sentence in the following way:
\begin{enumerate}[noitemsep,topsep=0.2cm]
\item  Annotate the verb with the relevant sense.
\item  Annotate all words in an argument with the relevant argument label.
\end{enumerate}
We will keep things simple and simply add the relevant label in the
last column of the CoNLL format.\footnote{In general, it is much
  better to use a dedicated annotation tool to prevent typographical
  errors when editing the file.}  Thus, here is what the annotation of
a hypothetical examle should look like:
\begin{Verbatim}[fontsize=\small,xleftmargin=\parindent]
1    We       we       PRON     PRON     _    2    nsubj    _    A0
2    count    count    VERB     VERB     _    0    root     _    count.03
3    on       on       ADP      ADP      _    4    case     _    A1
4    you      you      PRON     PRON     _    2    nmod     _    A1
5    .        .        PUNCT    PUNCT    _    0    punct    _    _
\end{Verbatim}
Note that your task is restricted to the annotation of predicate
senses (like {\tt count.03}) and core argument roles (numbered roles
like {\tt A0}, {\tt A1} and {\tt A2}). You do not need to annotate
modifier roles like {\tt AM-TMP} and {\tt AM-LOC}.

\newsec{Compare your annotations }{compare}%
Team up with a classmate (or two) and compare your respective
annotations. {\bf NB:} Make sure to save a copy of your original
annotation before starting to compare and discuss them.  Answer the
following questions:
\begin{enumerate}[noitemsep,topsep=0.2cm]
\item  What percentage of verb senses did you agree on? 
\item  What percentrage of argument {\em roles} did you agree on?
\item  What percentrage of argument {\em spans} did you agree on?
\end{enumerate}
After identifying all disagreements, your final task is to resolve
them: create a file annotated for the whole group.
%TODO: this is not very clear

\end{document}
