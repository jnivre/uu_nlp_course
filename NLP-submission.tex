%%%%%%%%%%%%%%%%%%%%%%%%%%%%%%%%%%%%%%%%%%%%%%%%%%%%%%%
%%% LATEX FORMATTING - LEAVE AS IS %%%%%%%%%%%%%%%%%%%%
\documentclass[11pt]{article} % documenttype: article
\usepackage[top=20mm,left=20mm,right=20mm,bottom=15mm,headsep=15pt,footskip=15pt,a4paper]{geometry} % customize margins
\usepackage{times} % fonttype
\makeatletter         
\def\@maketitle{   % custom maketitle 
\begin{center}
{\bfseries \@title}
{\bfseries \@author}
\end{center}
\smallskip \hrule \bigskip }

%%%%%%%%%%%%%%%%%%%%%%%%%%%%%%%%%%%%%%%%%%%%%%%%%%%%%%%%%%%%%%%%%%%%
%%% MAKE CHANGES HERE %%%%%%%%%%%%%%%%%%%%%%%%%%%%%%%%%%%%%%%%%%%%%%
\title{{\LARGE Natural Language Processing: Assignment X}\\[1.5mm]} % Replace 'X' by number of Assignment
\author{Firstname Lastname} % Replace 'Firstname Lastname' by your name.
\date{} % delete this line to display the current date

%%%%%%%%%%%%%%%%%%%%%%%%%%%%%%%%%%%%%%%%%%%%%%%%%%%%%%%%%%%%%%%%%%%%
%%% BEGIN DOCUMENT %%%%%%%%%%%%%%%%%%%%%%%%%%%%%%%%%%%%%%%%%%%%%%%%%
%%% From here on, edit document. Use sections, subsections, etc.
%%% to structure your answers.
\begin{document}
\maketitle

\section{Introduction}
This is an example file of how your submissions should be formatted.
You can either use \LaTeX to typeset your submissions (recommended) or
any other editing device, as long as the submission is as similar as
possible to this example file we gave you.

\section{\LaTeX}
Typesetting texts in \LaTeX is common practise in academic
writing. During the NLP-course you have the chance to develop routine
to write and submit your textual assignments in \LaTeX. You may use
the source file \emph{NLP-submission.tex} located in
\emph{/local/kurs/nlp/basic1} and edit it accordingly.

\section{Other}
If you decide not to typeset your submissions in \LaTeX, please make
sure they look similar to this example file. Particularly, please pay
attention to the margins and the font size. After having edited your
submission, please make sure to transform it to .pdf format before
submission. We will not accept any other format.

\end{document}