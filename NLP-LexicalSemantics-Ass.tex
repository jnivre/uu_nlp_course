\documentclass[11pt]{article}
%\usepackage[top=20mm,left=20mm,right=20mm,bottom=15mm,a4paper]{geometry} % see geometry.pdf on how to lay out the page. There's lots.
\usepackage[top=30mm,left=20mm,right=20mm,bottom=30mm,headsep=15pt,footskip=15pt,a4paper]{geometry} % see geometry.pdf on how to lay out the page. There's lots.
%\geometry{a4paper} % or letter or a5paper or ... etc
% \geometry{landscape} % rotated page geometry
\usepackage[round]{natbib}
\setlength{\bibsep}{0.0pt}
\usepackage{color}
\usepackage{times}
%\usepackage[T1]{fontenc}
%\usepackage{mathptmx}
\usepackage{tikz-dependency}
\usepackage{enumitem}
%\usepackage{times}

\usepackage[procnames]{listings}
\usepackage{color}
\usepackage{array}
 

% See the ``Article customise'' template for come common customisations
\newcommand{\refeq}[1]{Equation~\ref{eq:#1}}
\newcommand{\reffig}[1]{Figure~\ref{fig:#1}}
\newcommand{\reftab}[1]{Table~\ref{tab:#1}}
\newcommand{\refsec}[1]{\textsection\ref{sec:#1}}
\newcommand{\newsec}[2]{\section{#1}\label{sec:#2}\noindent}
\newcommand{\newsubsec}[2]{\subsection{#1}\label{sec:#2}\noindent}
\newcommand{\argmax}{\operatornamewithlimits{argmax}} 
\newcommand{\argmin}{\operatornamewithlimits{argmin}} 

\makeatletter         
\def\@maketitle{   % custom maketitle 
\begin{center}%
{\bfseries \@title}%
{\bfseries \@author}%
\end{center}%
\smallskip \hrule \bigskip }

% custom section 
\renewcommand{\section}{\@startsection
{section}%                   % the name
{1}%                         % the level
{0mm}%                       % the indent
{-0.8\baselineskip}%            % the before skip
{0.3\baselineskip}%          % the after skip
{\bfseries\large}}% the style

% custom subsection 
\renewcommand{\subsection}{\@startsection
{subsection}%                   % the name
{2}%                         % the level
{0mm}%                       % the indent
{-0.8\baselineskip}%            % the before skip
{0.3\baselineskip}%          % the after skip
{\bfseries\large}}% the style

\renewcommand{\paragraph}{%
  \@startsection{paragraph}{4}%
  {\z@}{1.5ex \@plus 1ex \@minus .2ex}{-1em}%
  {\normalfont\normalsize\bfseries}%
}\makeatother

%\title{{\LARGE Universal Parser (UP)}\\[-8mm]
%\includegraphics[height=8mm]{RUPA}~~~~~~~~~~~~~~~~~~~~~~~~~~~~~~~~~~~~~~~~~~~~~~~~~~~~~~~~~~~~~~~~~~~~~~~~~~~\includegraphics[height=8mm]{RUPA}}
\title{{\LARGE Natural Language Processing}\\[1.5mm]{\large Lab 11: Word Sense Disambiguation}}
\author{}
\date{} % delete this line to display the current date

%%% BEGIN DOCUMENT
\begin{document}

\definecolor{keywords}{RGB}{255,0,90}
\definecolor{comments}{RGB}{0,0,113}
\definecolor{red}{RGB}{160,0,0}
\definecolor{green}{RGB}{0,150,0}
 
\lstset{language=Python, 
        basicstyle=\ttfamily\small, 
        keywordstyle=\color{keywords},
        commentstyle=\color{comments},
        stringstyle=\color{red},
        showstringspaces=false,
        identifierstyle=\color{green},
        procnamekeys={def,class}}

\maketitle
%\tableofcontents
%\vspace{3mm}
\vspace{-2mm}
\newsec{Introduction}{intro}%
In this lab, we are going to explore the use of machine learning for word sense disambiguation.
More precisely, we will use NLTK to build Naive Bayes classifiers to disambiguate selected words
using data from Senseval-2. All files are available in {\tt /local/kurs/nlp/semantics/}.

\paragraph{Acknowledgment:} Thanks to Alex Lascarides for letting us reuse
and modify an older lab, including the code in {\tt wsd\_code.py}.

\newsec{Data}{data}%
The Senseval-2 corpus consists of text from a mixture of places,
including the British National Corpus and the Wall Street Journal
section of the Penn Treebank. Each word in the corpus is tagged with
its part of speech, and the senses of the following target words are
also annotated: the noun {\em interest}; the verb {\em serve}; and the
adjective {\em hard}. The senses used to annotate each target word
come from the Longman Dictionary of Contemporary English and WordNet
and are shown in the table below.\footnote{These are subsets of the
  full set of senses defined in WordNet, which is why the numbering of
  senses is not (always) contiguous.}

\begin{center}
\scalebox{0.8}{
\renewcommand{\arraystretch}{1.5}
\begin{tabular}{ll>{\raggedright}p{8.5cm}p{8cm}}
\textbf{Word} & \textbf{Sense} & \textbf{Definition} & \textbf{Example} \\
\hline
interest &interest$_1$ &readiness to give attention&``an interest in music" \\
&interest$_2$ &quality of causing attention to be given to&``they said nothing of great interest" \\
&interest$_3$ &activity, etc. that one gives attention to&``she has many interests'' \\
&interest$_4$ &advantage, advancement or favor & ``in the interest of safety" \\
&interest$_5$ &a share in a company or business&``they have interests all over the world" \\
&interest$_6$ &money paid for the use of money&``how much interest do you pay on your mortgage?"\\
\hline
hard &hard$_1$ &not easy; requiring great physical or mental effort to accomplish or comprehend or endure &``a hard task" \\
&hard$_2$ &dispassionate &``a hard bargainer" \\
&hard$_3$ &resisting weight or pressure &``a hard rock" \\
\hline
serve &serve$_2$ &do duty or hold offices; serve in a specific function & ``she served in Congress for two terms" \\
&serve$_6$ &provide (usually but not necessarily) food & ``we serve meals for the homeless" \\
&serve$_{10}$ &work for or be a servant to & ``may I serve you?" \\
&serve$_{12}$ &be sufficient; be adequate, either in quality or quantity & ``nothing else will serve" \\
\end{tabular}
}
\end{center}
Let us now start exploring the sense-tagged data using the Python interpreter. Copy the file {\tt wsd\_code.py} from {\tt /local/kurs/nlp/semantics/} to your working directory and start the Python interpreter with the command {\tt python}.
\begin{center}
\fbox{
\scalebox{0.95}{
\lstinputlisting{code/explore_senses.py}
}}
\end{center}
The session above shows how to use the method {\tt senses()} to list the possible senses of the word \emph{hard} and how
to retrieve the set of all sense-tagged instances of this word. After using the method {\tt instances()}, the value of {\tt hard} is a list containing all the instances of \emph{hard} in the corpus, and we can retrieve the first instance as {\tt hard[0]}. An instance has
four different attributes:
\begin{enumerate}[topsep=0.2cm,noitemsep]
\item {\tt word} specifies the target word together with its syntactic category; for example,  {\tt hard-a} 
means that the word is {\em hard} and that its category is {\em adjective};
\item {\tt position} gives its position within the sentence (starting with position 0);
\item {\tt context} represents the sentence as a list of pairs, each consisting of a token and its tag; 
\item {\tt senses} is a tuple, each item in the tuple being a sense for the target word; typically, this tuple consists of only one argument, 
but there are a few examples in the corpus where there is more than one, representing the fact that the annotator couldn't decide which sense to assign to the word; for simplicity, we are going to ignore any non-first arguments to the attribute {\tt senses}
\end{enumerate}
Make sure you understand the representation of instances. Look at a few more instances of \emph{hard} and then do 
the same for \emph{interest} and \emph{serve} (by replacing {\tt 'hard.pos'} with {\tt 'interest.pos'} and {\tt 'serve.pos'}, respectively).

\newsec{Baseline classifiers }{baseline}%
Before we build sense classifiers for \emph{interest}, \emph{hard} and
\emph{serve}, we need to establish baselines for comparison.  Two
commonly used baselines are the \emph{random} baseline and the
\emph{majority} baseline. The random baseline is the accuracy that we
can expect from guessing senses randomly.
\begin{enumerate}[topsep=0.2cm,noitemsep]
\item  What is the random baseline for \emph{hard} given our current
  sense inventory?
\item  Is the random baseline the same for \emph{interest} and
  \emph{serve}? Why (not)?
\end{enumerate}
The majority baseline is the accuracy we obtain if we always guess the
most frequent sense. We can derive the majority baseline for
\emph{hard} using NLTK methods as follows:
\begin{center}
\fbox{
\scalebox{0.9}{
\lstinputlisting{code/majority_baseline.py}
}}
\end{center}
We use the method {\tt nltk.FreqDist()} to compute the distribution of
different senses of \emph{hard} in the Senseval corpus. The output is
stored in the variable {\tt hard\_dist} and shows that {\tt 'HARD1'}
has 3455 occurrences, while {\tt 'HARD2'} and {\tt 'HARD3'} occur only
502 and 376 times, respectively. Hence, by always guessing that
\emph{hard} has the first sense, we reach an accuracy (on this sample)
equal to 3455 / (3455 + 502 + 376) = 0.797, or 79.7\%.
\begin{enumerate}[topsep=0.2cm,noitemsep]
\setcounter{enumi}{2}
\item Compute the majority baselines for \emph{interest} and \emph{serve}, respectively.
\end{enumerate}

\newsec{Naive Bayes classifiers }{naive}%
We are now going to compare the performance of different classifiers
that perform word sense disambiguation, using the method {\tt
  wsd\_classifier()}, which must have at least the following arguments
specified by you:
\begin{enumerate}[topsep=0.2cm,noitemsep]
\item a trainer, in this case {\tt NaiveBayesClassifier.train};
\item a target word: {\tt 'hard.pos'}, {\tt 'interest.pos'} or {\tt
    'serve.pos'};
\item one of two feature representations:
\begin{itemize}[noitemsep,topsep=0.1cm]
\item {\tt wsd\_word\_features}: a bag-of-words representation defined
  over the 300 most frequent words that co-occur with the target word
  in the training corpus (after filtering out stop words);
\item {\tt wsd\_context\_features}: a collocational feature
  representation consisting of 3 preceding and 3 succeeding word-tag
  pairs;
\end{itemize}
\end{enumerate}
The method splits the data available for a target word into a {\em
  training set} and a {\em test set}, trains a Naive Bayes classifier
on the training set, and evaluates its accuracy on the test set. Here
is how you use the method to evaluate the bag-of-words model for the
target word \emph{hard}:
\begin{center}
\fbox{
\scalebox{0.8}{
\lstinputlisting{code/naive_bayes.py}
}}
\end{center}
You should do the following:
\begin{enumerate}[noitemsep,topsep=0.2cm]
\item  Evaluate both feature representations for all three target words.
\item  Compare the results for different classifiers, including your baselines.
\end{enumerate}

%TODO: move this to the assignment
%TODO: add exercise about precision and recall to lab
% \newsec{Error analysis }{submit}%
% After analyzing the performance of a classifier, it is useful to
% perform an error analysis. By providing additional arguments to {\tt
%   wsd\_classifier()}, you can get a confusion matrix as well as a
% printout of all the errors.
% \begin{center}
% \fbox{
% \scalebox{0.55}{
% \lstinputlisting{code/error.py}
% }}
% \end{center}
% The relevant arguments are {\tt confusion\_matrix=True} and {\tt
%   log=True}, but note that you must also insert the argument {\tt
%   distance=3} to get the right interpretation. The errors are printed
% to an external file called {\tt errors.txt}. Do an error analysis of
% the best performing classifier for each target word, answering the
% following questions:
% \begin{enumerate}
% \item  Using the confusion matrix, identify which sense is the hardest
%   one for the model to estimate.
% \item  Look in {\tt errors.txt} for examples where that hardest word
%   sense is the correct label.  Do you see any patterns or systematic
%   errors? If so, can you think of a way to adapt the feature
%   representation to improve the model?
% \end{enumerate}
% \newsec{Submit the assignment}{submit}%
% Upload the following to {\it Studentportalen} \textbf{before 20:00h
%   December 16th}. In order to pass you must provide answers to all
% questions and achieve at least 15/20 points.
% \begin{itemize}[noitemsep,topsep=0.2cm]
% \item Submission format: \texttt{.pdf} format written in \LaTeX
% \item Answers to the questions in Section 3 
% \item A table summarizing the results for four classifiers (Section 3
%   and 4: random, majority, bag-of-words, collocation) and three target
%   words (\emph{hard}, \emph{interest}, \emph{serve}).
% \item A short discussion of the results (Section 4, max 1/2 page)
% \item An error analysis for the best classifier for each target word (Section 5)
% \item \textbf{[VG]}: In order to get the grade VG, reach at least 18
%   points in the exercises above.
% \end{itemize}

\end{document}
