
\documentclass[11pt]{article}
%\usepackage[top=20mm,left=20mm,right=20mm,bottom=15mm,a4paper]{geometry} % see geometry.pdf on how to lay out the page. There's lots.
\usepackage[top=20mm,left=20mm,right=20mm,bottom=15mm,headsep=15pt,footskip=15pt,a4paper]{geometry} % see geometry.pdf on how to lay out the page. There's lots.
%\geometry{a4paper} % or letter or a5paper or ... etc
% \geometry{landscape} % rotated page geometry
\usepackage[round]{natbib}
\setlength{\bibsep}{0.0pt}
\usepackage{color}
\usepackage{times}
%\usepackage[T1]{fontenc}
%\usepackage{mathptmx}
\usepackage{tikz-dependency}
\usepackage{enumitem}
%\usepackage{times}

\usepackage[procnames]{listings}
\usepackage{color}
 
 

% See the ``Article customise'' template for come common customisations
\newcommand{\refeq}[1]{Equation~\ref{eq:#1}}
\newcommand{\reffig}[1]{Figure~\ref{fig:#1}}
\newcommand{\reftab}[1]{Table~\ref{tab:#1}}
\newcommand{\refsec}[1]{\textsection\ref{sec:#1}}
\newcommand{\newsec}[2]{\section{#1}\label{sec:#2}\noindent}
\newcommand{\newsubsec}[2]{\subsection{#1}\label{sec:#2}\noindent}
\newcommand{\argmax}{\operatornamewithlimits{argmax}} 
\newcommand{\argmin}{\operatornamewithlimits{argmin}} 

\makeatletter         
\def\@maketitle{   % custom maketitle 
\begin{center}%
{\bfseries \@title}%
{\bfseries \@author}%
\end{center}%
\smallskip \hrule \bigskip }

% custom section 
\renewcommand{\section}{\@startsection
{section}%                   % the name
{1}%                         % the level
{0mm}%                       % the indent
{-0.8\baselineskip}%            % the before skip
{0.3\baselineskip}%          % the after skip
{\bfseries\large}}% the style

% custom subsection 
\renewcommand{\subsection}{\@startsection
{subsection}%                   % the name
{2}%                         % the level
{0mm}%                       % the indent
{-0.8\baselineskip}%            % the before skip
{0.3\baselineskip}%          % the after skip
{\bfseries\large}}% the style

\renewcommand{\paragraph}{%
  \@startsection{paragraph}{4}%
  {\z@}{1.5ex \@plus 1ex \@minus .2ex}{-1em}%
  {\normalfont\normalsize\bfseries}%
}\makeatother

%\title{{\LARGE Universal Parser (UP)}\\[-8mm]
%\includegraphics[height=8mm]{RUPA}~~~~~~~~~~~~~~~~~~~~~~~~~~~~~~~~~~~~~~~~~~~~~~~~~~~~~~~~~~~~~~~~~~~~~~~~~~~\includegraphics[height=8mm]{RUPA}}
\title{{\LARGE Natural Language Processing}\\[1.5mm]{\large Assignment 9: Constituency Parsing}}
\author{}
\date{} % delete this line to display the current date

%%% BEGIN DOCUMENT
\begin{document}

\definecolor{keywords}{RGB}{255,0,90}
\definecolor{comments}{RGB}{0,0,113}
\definecolor{red}{RGB}{160,0,0}
\definecolor{green}{RGB}{0,150,0}
 
\lstset{language=Python, 
        basicstyle=\ttfamily\small, 
        keywordstyle=\color{keywords},
        commentstyle=\color{comments},
        stringstyle=\color{red},
        showstringspaces=false,
        identifierstyle=\color{green},
        procnamekeys={def,class}}

\maketitle
%\tableofcontents
%\vspace{3mm}
\vspace{-2mm}
\newsec{Introduction}{intro}%
In the previous assignment, we extracted a probabilistic context-free grammar from the first 1000 sentences of the WSJ section of the Penn Treebank.
In this assignment, we are going to put the grammar to use for syntactic parsing. We will continue to use NLTK. 
Additional files can be found in {\tt /local/kurs/nlp/syntax/}.

\newsec{Parse a sentence (6 P)}{chart}%
At the end of the previous assignment, you were asked to guess the number of parse trees generated by the grammar for the sentence: ``The rates rise .'' 
In order to find out the answer to this question, we are going to parse the sentence using our treebank grammar and a non-probabilistic parser that derives
all possible parse trees for an input sentence.
\begin{center}
\fbox{
\lstinputlisting{code/chart_parse.py}
}
\end{center}
The above listing shows how to extract the grammar (just like last time), how to create a chart parser for that grammar using NLTK,
and how to parse the sentence.\footnote{Note that the input is a string split into a list of tokens, equivalent to {\tt c.parse(['The', 'rates', 'rise', '.'])}.} 
After executing the parse command again, you can print the parse trees one by one using the command {\tt print(p.next())}.
\begin{center}
\fbox{
\lstinputlisting{code/print_parses.py}
}
\end{center}
{\bf Questions:} 
\begin{enumerate}[noitemsep,topsep=0.2cm]
\item (2 P) How many of the parse trees look reasonable?
\item (2 P) Why (not)?
\item (2 P) Can you find the correct parse tree?
\end{enumerate}

\newsec{Find the most probable parse (6 P)}{viterbi}%
The parser used so far just returns all possible parse trees without ranking them, which is normally not very useful.  
A probabilistic parser returns the parse tree that has the highest probability according to the grammar. To find the most
probable parse tree, you can use the {\tt ViterbiParser} from {\tt nltk.parse}. 
\begin{center}
\fbox{
\lstinputlisting{code/viterbi_parse.py}
}
\end{center}
{\bf Questions:}
\begin{enumerate}[noitemsep,topsep=0.2cm]
\item (2 P) Does the most probable parse tree look reasonable? 
\item (2 P) Why (not)? 
\item (2 P) Is it the correct parse tree?
\end{enumerate}

\clearpage
\newsec{Parse some more sentences (8 P)}{web}%
Try to parse the following sentences with the probabilistic parser:
\begin{enumerate}[noitemsep,topsep=0.2cm]
\item (1 P) ``Pierre Vinken is a director .''\label{first}
\item (2 P) ``Pierre Vinken is a fool .''
\item (2 P) ``Pierre Vinken hit the ball with the club .''
\item (3 P) ``Pierre Vinken saw the man with the club .''\label{last}
\end{enumerate}

\noindent
{\bf Question:}\\
- Are you happy with the most probable parse tree for \ref{first}--\ref{last}? Discuss!

\newsec{Submit the assignment}{sub}%
Submit the following to {\tt nlp-course@stp.lingfil.uu.se} \textbf{before 20:00h December 9th}. In order to pass you must provide answers to all questions and achieve at least 15/20 points.  
\begin{itemize}[noitemsep,topsep=0.2cm]
\item Well motivated answers to the questions in Section~\ref{sec:chart}--\ref{sec:web}.
\item \textbf{[VG]} ??
\end{itemize}

\end{document}