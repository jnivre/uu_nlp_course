
\documentclass[11pt]{article}
%\usepackage[top=20mm,left=20mm,right=20mm,bottom=15mm,a4paper]{geometry} % see geometry.pdf on how to lay out the page. There's lots.
\usepackage[top=20mm,left=20mm,right=20mm,bottom=15mm,headsep=15pt,footskip=15pt,a4paper]{geometry} % see geometry.pdf on how to lay out the page. There's lots.
%\geometry{a4paper} % or letter or a5paper or ... etc
% \geometry{landscape} % rotated page geometry
\usepackage[round]{natbib}
\setlength{\bibsep}{0.0pt}
\usepackage{color}
\usepackage{times}
%\usepackage[T1]{fontenc}
%\usepackage{mathptmx}
\usepackage{tikz-dependency}
\usepackage{enumitem}
%\usepackage{times}

\usepackage[procnames]{listings}
\usepackage{color}
 
 

% See the ``Article customise'' template for come common customisations
\newcommand{\refeq}[1]{Equation~\ref{eq:#1}}
\newcommand{\reffig}[1]{Figure~\ref{fig:#1}}
\newcommand{\reftab}[1]{Table~\ref{tab:#1}}
\newcommand{\refsec}[1]{\textsection\ref{sec:#1}}
\newcommand{\newsec}[2]{\section{#1}\label{sec:#2}\noindent}
\newcommand{\newsubsec}[2]{\subsection{#1}\label{sec:#2}\noindent}
\newcommand{\argmax}{\operatornamewithlimits{argmax}} 
\newcommand{\argmin}{\operatornamewithlimits{argmin}} 

\makeatletter         
\def\@maketitle{   % custom maketitle 
\begin{center}%
{\bfseries \@title}%
{\bfseries \@author}%
\end{center}%
\smallskip \hrule \bigskip }

% custom section 
\renewcommand{\section}{\@startsection
{section}%                   % the name
{1}%                         % the level
{0mm}%                       % the indent
{-0.8\baselineskip}%            % the before skip
{0.3\baselineskip}%          % the after skip
{\bfseries\large}}% the style

% custom subsection 
\renewcommand{\subsection}{\@startsection
{subsection}%                   % the name
{2}%                         % the level
{0mm}%                       % the indent
{-0.8\baselineskip}%            % the before skip
{0.3\baselineskip}%          % the after skip
{\bfseries\large}}% the style

\renewcommand{\paragraph}{%
  \@startsection{paragraph}{4}%
  {\z@}{1.5ex \@plus 1ex \@minus .2ex}{-1em}%
  {\normalfont\normalsize\bfseries}%
}\makeatother

%\title{{\LARGE Universal Parser (UP)}\\[-8mm]
%\includegraphics[height=8mm]{RUPA}~~~~~~~~~~~~~~~~~~~~~~~~~~~~~~~~~~~~~~~~~~~~~~~~~~~~~~~~~~~~~~~~~~~~~~~~~~~\includegraphics[height=8mm]{RUPA}}
\title{{\LARGE Natural Language Processing}\\[1.5mm]{\large Assignment
    0: \LaTeX}} \author{}
\date{} % delete this line to display the current date

%%% BEGIN DOCUMENT
\begin{document}

\definecolor{keywords}{RGB}{255,0,90}
\definecolor{comments}{RGB}{0,0,113}
\definecolor{red}{RGB}{160,0,0}
\definecolor{green}{RGB}{0,150,0}
 
\lstset{language=Python, 
        basicstyle=\ttfamily\small, 
        keywordstyle=\color{keywords},
        commentstyle=\color{comments},
        stringstyle=\color{red},
        showstringspaces=false,
        identifierstyle=\color{green},
        procnamekeys={def,class}}

\maketitle
%\tableofcontents
%\vspace{3mm}
\vspace{-2mm} \newsec{Introduction}{intro}%
Typesetting texts in \LaTeX is common practise in academic
writing. During the NLP-course you will develop routine to write and
submit your textual assignments in \LaTeX. For that, we provide an
empty \LaTeX file (\texttt{submission.tex}) in the
\texttt{/local/kurs/nlp/basic1/latex/} folder, which you can copy and
use for your submissions. If you have never used \LaTeX before, take a
look at the tutorial \texttt{latex\_tutorial.pdf} to be found in the
same folder. \LaTeX is free and open-source and very well documented
in several books, but even online. You can use it on the Linux command
line (as explained in the tutorial) or install an \LaTeX compiler on
your local computer.

\newsec{Submit the assignment}{submit}%
Upload the following to {\it Studentportalen} {\bf before 20:00h November 9th}: 
\begin{itemize}[noitemsep,topsep=0.2cm]
\item A compiled .pdf document based on \texttt{submission.tex} with
  your name on it. 
\item Note that Assignment 1 is not to be submitted in \texttt{.pdf},
  but in plain \texttt{.txt} format. For each part of the assignments
  we will clearly indicate the file format for submission.
\end{itemize}

%\bigskip
%\begin{center}
%{\huge {\bf NLP}}\\
%\fbox{
%\includegraphics[scale=0.2]{nlp}
%}
%\end{center}

\end{document}