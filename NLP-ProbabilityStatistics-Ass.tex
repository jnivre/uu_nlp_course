
\documentclass[11pt]{article}
%\usepackage[top=20mm,left=20mm,right=20mm,bottom=15mm,a4paper]{geometry} % see geometry.pdf on how to lay out the page. There's lots.
\usepackage[top=20mm,left=20mm,right=20mm,bottom=15mm,headsep=15pt,footskip=15pt,a4paper]{geometry} % see geometry.pdf on how to lay out the page. There's lots.
%\geometry{a4paper} % or letter or a5paper or ... etc
% \geometry{landscape} % rotated page geometry
\usepackage[round]{natbib}
\setlength{\bibsep}{0.0pt}
\usepackage{color}
\usepackage{times}
%\usepackage[T1]{fontenc}
%\usepackage{mathptmx}
\usepackage{tikz-dependency}
\usepackage{enumitem}
%\usepackage{times}

\usepackage[procnames]{listings}
\usepackage{color}
\usepackage{todonotes}
 
 

% See the ``Article customise'' template for come common customisations
\newcommand{\refeq}[1]{Equation~\ref{eq:#1}}
\newcommand{\reffig}[1]{Figure~\ref{fig:#1}}
\newcommand{\reftab}[1]{Table~\ref{tab:#1}}
\newcommand{\refsec}[1]{\textsection\ref{sec:#1}}
\newcommand{\newsec}[2]{\section{#1}\label{sec:#2}\noindent}
\newcommand{\newsubsec}[2]{\subsection{#1}\label{sec:#2}\noindent}
\newcommand{\argmax}{\operatornamewithlimits{argmax}} 
\newcommand{\argmin}{\operatornamewithlimits{argmin}} 

\makeatletter         
\def\@maketitle{   % custom maketitle 
\begin{center}%
{\bfseries \@title}%
{\bfseries \@author}%
\end{center}%
\smallskip \hrule \bigskip }

% custom section 
\renewcommand{\section}{\@startsection
{section}%                   % the name
{1}%                         % the level
{0mm}%                       % the indent
{-0.8\baselineskip}%            % the before skip
{0.3\baselineskip}%          % the after skip
{\bfseries\large}}% the style

% custom subsection 
\renewcommand{\subsection}{\@startsection
{subsection}%                   % the name
{2}%                         % the level
{0mm}%                       % the indent
{-0.8\baselineskip}%            % the before skip
{0.3\baselineskip}%          % the after skip
{\bfseries\large}}% the style

\renewcommand{\paragraph}{%
  \@startsection{paragraph}{4}%
  {\z@}{1.5ex \@plus 1ex \@minus .2ex}{-1em}%
  {\normalfont\normalsize\bfseries}%
}\makeatother

%\title{{\LARGE Universal Parser (UP)}\\[-8mm]
%\includegraphics[height=8mm]{RUPA}~~~~~~~~~~~~~~~~~~~~~~~~~~~~~~~~~~~~~~~~~~~~~~~~~~~~~~~~~~~~~~~~~~~~~~~~~~~\includegraphics[height=8mm]{RUPA}}
\title{{\LARGE Natural Language Processing}\\[1.5mm]{\large Lab 3: Maximum Likelihood Estimation}}
\author{}
\date{} % delete this line to display the current date

%%% BEGIN DOCUMENT
\begin{document}

\definecolor{keywords}{RGB}{255,0,90}
\definecolor{comments}{RGB}{0,0,113}
\definecolor{red}{RGB}{160,0,0}
\definecolor{green}{RGB}{0,150,0}
 
\lstset{language=Python, 
        basicstyle=\ttfamily\small, 
        keywordstyle=\color{keywords},
        commentstyle=\color{comments},
        stringstyle=\color{red},
        showstringspaces=false,
        identifierstyle=\color{green},
        procnamekeys={def,class}}

\maketitle
%\tableofcontents
%\vspace{3mm}
\vspace{-2mm} \newsec{Introduction}{intro}%
In this lab, we are going to start using statistical inference
to build probability models from data.  We will use an English text
corpus to estimate bigram probabilities and then use probability
theory to derive various related probabilities. Next time, we will
explore the use of $n$-gram models for the problem of language
modeling. All the files needed for the lab are in {\tt
  /local/kurs/nlp/ngram/}. Make copies of the files into your home
directory (as you did in Labs 1 and 2).

\newsec{Data and preprocessing}{data}%
Our statistical sample will be three collections of short stories
about Sherlock Holmes by Arthur Conan Doyle,\footnote{The collections
  are {\em The Adventures of Sherlock Holmes}, {\em The Memoirs of
    Sherlock Holmes}, and {\em The Return of Sherlock Holmes}.}
available as plain text in the file {\tt holmes.txt}. We will start by
tokenizing the text and segmenting it into sentences.  You could in
principle use your own tokenizer, but to make sure that everyone has
the same version of the text it is better to use {\tt tokenizer5.py},
which takes some special properties of this text into account.

\newsec{Extract bigram frequencies }{bigrams}%
Estimation of probabilities is always based on frequency data, and we
will start by computing the frequency of \emph{word bigrams} in our
corpus. This is what the Python program {\tt bigrams.py} does.

\begin{center}
\fbox{
\lstinputlisting{code/bigrams.py}
}
\end{center}
\newpage
\noindent
The program reads from {\tt stdin} and prints to {\tt stdout}, so in
order to count the bigrams in the text you need to run something like:
\begin{verbatim}
    python bigrams.py < holmes-tokens.txt > holmes-bigrams.txt
\end{verbatim}
The output consists of all bigrams found in the text, sorted
alphabetically with one bigram on each line, followed by its
frequency.

%\paragraph{Note on programming:} 
%If you don't want to use {\tt bigrams.py}, you can write your own program, 
%or you can use standard linux/unix commands like {\tt paste}, {\tt sort} and {\tt unix}, 
%as described in Ken Church's classic tutorial {\em Unix for Poets}.%
%\footnote{http://stp.lingfil.uu.se/~ch/ist2014/UnixforPoets.pdf}

\paragraph{Note on sentence boundaries:} If you look closely at {\tt
  bigrams.py} (or its output), you can see that the two special tokens
{\tt <s>} and {\tt <e>} are inserted at sentence boundaries to mark
the {\em start} and {\em end} of a sentence, respectively. This is
necessary to get a properly normalized language model eventually, but
for now you just need to be aware of it.

\vspace{0.3cm}
\noindent
Have a look at the bigram frequencies derived from the Holmes corpus
and try to answer the following questions before you go on to the next
section:
\begin{enumerate}
\item  How many bigram {\em tokens} are there in the corpus? % wc holmes-tok.txt = 132961
How many bigram {\em types}? % wc holmes-bi.txt = 50132
\item  What is the most frequent bigram in the corpus? % sort -k 3 -r -n holmes-bi.txt = . <e>
What is the most frequent bigram that does not involve punctuation or one of the special tokens {\tt <s>} and {\tt <e>}? % of the
\item  What is the frequency of the bigram ``Sherlock Holmes''? % grep 'Sherlock.Holmes' holmes-bi.txt = 94
What about ``Holmes Sherlock''? % grep 'Holmes.Sherlock' holmes-bi.txt = 0
\item  How many bigrams only occur once? % grep '[^0-9]1$' holmes-bi.txt | wc = 36763
\item  How many bigrams do not occur at all (assuming that the entire vocabulary is made up of tokens that occur at least once in the corpus)?
% grep -v '^$' holmes-tok.txt | sort | uniq | wc = 8806 ; (8806 + 2)^2 - (2 * 8808) = 77 563 248
\end{enumerate}

\paragraph{Note on linux commands:} The frequencies needed to answer
these questions can be computed by a modified version of {\tt
  bigrams.py}. In most cases, however, it is easier to just use
standard unix/linux commands like {\tt wc}, {\tt grep}, and {\tt
  sort}.  If you are unsure how to do this, check out
Appendix~\ref{sec:unix}.

\newsec{Maximum likelihood estimation }{mle}%
The next step is to use the frequency counts to estimate a probability
distribution over bigrams, that is, the joint distribution $P(w_1,
w_2)$, where $w_1$ and $w_2$ are the first and second word of a
bigram, respectively.  We will use the maximum likelihood estimate
(MLE). Start by discussing the following questions:
\begin{enumerate}[itemsep=0pt]
\item  What does it mean that an estimate is MLE? % Maximizes probability of corpus
\item  Let $|W|$ be the number of words in our vocabulary, and let $P(w_1, w_2) = \frac{1}{|W|^2}$ for all $(w_1, w_2)$.\\
  Why is this not an MLE for our
  corpus? % Too much probability to rare bigrams, not enough to frequent bigrams
\item  What is the MLE of $P(\mbox{Sherlock}_1, \mbox{Holmes}_2)$ for
  our corpus? % 94 / 132961
\item  What is the MLE of $P(\mbox{Holmes}_1, \mbox{Sherlock}_2)$? % 0
\end{enumerate}
When you have convinced yourselves that you can answer these
questions, go ahead and compute the MLE for all word bigrams occurring
in our corpus. For this you need to modify {\tt bigrams.py} so that it
prints the bigram probability estimate instead of the bigram frequency
as the last element on each line .

\newsec{More probability estimates }{refine}%
Given your estimates of the joint bigram probabilities, you should be
able to derive estimates for the following:
\begin{enumerate}[itemsep=0pt]
\item  The marginal probability $P(w_1)$
\item  The conditional probability $P(w_2 | w_1)$
\end{enumerate}

\appendix
\newsec{Useful unix/linux commands}{unix}%
\begin{itemize}
\item To count the number of lines, words, and characters in a file,
  use {\tt wc}:
\begin{verbatim}
prompt$ wc holmes-tokens.txt
  132961  126113  587964 holmes-tokens.txt
\end{verbatim}
This tells us that {\tt holmes-tokens.txt} consists of 132691 lines, 126113 words, 587964 characters. 

\item To sort the lines of a file, use {\tt sort}. The flag {\tt -k} can be used to specify which column should be used 
for sorting and the flag {\tt -n} makes the sorting numerical rather than alphabetical. 
Thus, the following command sorts the bigram file with respect to frequency:
\begin{verbatim}
prompt$ sort -k 3 -n holmes-bigrams.txt
\end{verbatim}
The flag {\tt -r} reverses the sort order (in this case to decreasing frequency instead of increasing).
\item To find all lines that match a given regular expression, use {\tt grep}. For example, this commands finds all the lines
in the bigram file with a frequency of 10:
\begin{verbatim}
prompt$ grep '[^0-9]10$' holmes-bigrams.txt
\end{verbatim}
The flag {\tt -v} inverts the search to match all lines \textbf{not} matching the pattern. Thus, the following command
finds all non-empty lines in a file:
\begin{verbatim}
prompt$ grep -v '^$' holmes-bigrams.txt
\end{verbatim}
\item To count the number of unique lines (types rather than tokens), combine {\tt sort}, {\tt uniq} and {\tt wc} as follows:
\begin{verbatim}
prompt$ sort holmes-tokens.txt | uniq | wc
    8807    8806   70763
\end{verbatim}
The {\tt uniq} command removes adjacent repetitions of the same line, and the pipe symbol {\tt |} is used to 
pass the output of one command as input to the next.
\end{itemize}

\end{document}
