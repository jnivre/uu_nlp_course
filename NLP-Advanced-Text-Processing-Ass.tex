
\documentclass[11pt]{article}
%\usepackage[top=20mm,left=20mm,right=20mm,bottom=15mm,a4paper]{geometry} % see geometry.pdf on how to lay out the page. There's lots.
\usepackage[top=20mm,left=20mm,right=20mm,bottom=15mm,headsep=15pt,footskip=15pt,a4paper]{geometry} % see geometry.pdf on how to lay out the page. There's lots.
%\geometry{a4paper} % or letter or a5paper or ... etc
% \geometry{landscape} % rotated page geometry
\usepackage[round]{natbib}
\setlength{\bibsep}{0.0pt}
\usepackage{color}
\usepackage{times}
%\usepackage[T1]{fontenc}
%\usepackage{mathptmx}
\usepackage{tikz-dependency}
\usepackage{enumitem}
%\usepackage{times}

\usepackage[procnames]{listings}
\usepackage{color}
 
 

% See the ``Article customise'' template for come common customisations
\newcommand{\refeq}[1]{Equation~\ref{eq:#1}}
\newcommand{\reffig}[1]{Figure~\ref{fig:#1}}
\newcommand{\reftab}[1]{Table~\ref{tab:#1}}
\newcommand{\refsec}[1]{\textsection\ref{sec:#1}}
\newcommand{\newsec}[1]{\section{#1}\noindent}
%\newcommand{\newsec}[2]{\section{#1}\label{sec:#2}\noindent}
\newcommand{\newsubsec}[2]{\subsection{#1}\label{sec:#2}\noindent}
\newcommand{\argmax}{\operatornamewithlimits{argmax}} 
\newcommand{\argmin}{\operatornamewithlimits{argmin}} 

\makeatletter         
\def\@maketitle{   % custom maketitle 
\begin{center}%
{\bfseries \@title}%
{\bfseries \@author}%
\end{center}%
\smallskip \hrule \bigskip }

% custom section 
\renewcommand{\section}{\@startsection
{section}%                   % the name
{1}%                         % the level
{0mm}%                       % the indent
{-0.8\baselineskip}%            % the before skip
{0.3\baselineskip}%          % the after skip
{\bfseries\large}}% the style

% custom subsection 
\renewcommand{\subsection}{\@startsection
{subsection}%                   % the name
{2}%                         % the level
{0mm}%                       % the indent
{-0.8\baselineskip}%            % the before skip
{0.3\baselineskip}%          % the after skip
{\bfseries\large}}% the style

\renewcommand{\paragraph}{%
  \@startsection{paragraph}{4}%
  {\z@}{1.5ex \@plus 1ex \@minus .2ex}{-1em}%
  {\normalfont\normalsize\bfseries}%
}\makeatother

%\title{{\LARGE Universal Parser (UP)}\\[-8mm]
%\includegraphics[height=8mm]{RUPA}~~~~~~~~~~~~~~~~~~~~~~~~~~~~~~~~~~~~~~~~~~~~~~~~~~~~~~~~~~~~~~~~~~~~~~~~~~~\includegraphics[height=8mm]{RUPA}}
\title{{\LARGE Natural Language Processing}\\[1.5mm]{\large Assignment 2: Advanced Text Processing\\Lemmatisation and Part-of-Speech-Tagging}}
\author{}
\date{} % delete this line to display the current date

%%% BEGIN DOCUMENT
\begin{document}

\definecolor{keywords}{RGB}{255,0,90}
\definecolor{comments}{RGB}{0,0,113}
\definecolor{red}{RGB}{160,0,0}
\definecolor{green}{RGB}{0,150,0}
 
\lstset{language=Python, 
        basicstyle=\ttfamily\small, 
        keywordstyle=\color{keywords},
        commentstyle=\color{comments},
        stringstyle=\color{red},
        showstringspaces=false,
        identifierstyle=\color{green},
        procnamekeys={def,class}}

\maketitle
%\tableofcontents
%\vspace{3mm}

\newsec{Lemmatisation}%
%What are they supposed to do?

\begin{itemize}
\item Reflect on the importance of lemmatisation for your native
  language and at least one foreign language you know.\\
  \textbf{\textcolor{blue}{$\rightarrow$ [productivity $|$
      morphological richness]}}
\item How difficult do you estimate lemmatisation to be for your
  native language (and why)? \\\textbf{\textcolor{blue}{$\rightarrow$
      [ambiguity]}}
\item What is an FST? What is it used for and why is it useful? (ca. 1/2 page)\\
  \textbf{\textcolor{blue}{$\rightarrow$ [Technical background $|$
      Read the book, esp. 3.4-3.7]}}
\end{itemize}

\newsec{POS-Tagging}%

\begin{itemize}
\item What tagsets exist in your native language? List the ones you
  can find and describe one of them (5-10
  sentences). \\\textbf{\textcolor{blue}{[tagsets $|$ own web search]}}
\item Is it neccessary to tokenize text before tagging it? Please
  motivate your answer and give an example. (5-10
  sentences). \\\textbf{\textcolor{blue}{[Read the book]}}
\item In the HMM lab we have investigated key sequences and produced
  words. What do these correspond to when using HMMs for POS-tagging?
  \\\textbf{\textcolor{blue}{[HMMs $|$ Back-ref to lab]}}
\end{itemize}



\newsec{Summing up}%

\begin{itemize}
\item Why is it more difficult to tag morphologically rich
  languages?\\\textbf{\textcolor{blue}{[Morphological Productivity
      $\rightarrow$ Unknown words in tag lexicon]}}
\item What are the possible advantages/disadvantages of first
  performing lemmatisation and then POS-tagging and doing it vice
  versa? \\\textbf{\textcolor{blue}{$\rightarrow$ [error propagation
      $|$ reduce/enhance ambiguity $|$ out-of-the-box thinking]}}
\end{itemize}

\end{document}