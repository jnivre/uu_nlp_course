
\documentclass[11pt]{article}
%\usepackage[top=20mm,left=20mm,right=20mm,bottom=15mm,a4paper]{geometry} % see geometry.pdf on how to lay out the page. There's lots.
\usepackage[top=20mm,left=20mm,right=20mm,bottom=15mm,headsep=15pt,footskip=15pt,a4paper]{geometry} % see geometry.pdf on how to lay out the page. There's lots.
%\geometry{a4paper} % or letter or a5paper or ... etc
% \geometry{landscape} % rotated page geometry
\usepackage[round]{natbib}
\setlength{\bibsep}{0.0pt}
\usepackage{color}
\usepackage{times}
%\usepackage[T1]{fontenc}
%\usepackage{mathptmx}
\usepackage{tikz-dependency}
\usepackage{enumitem}
%\usepackage{times}

\usepackage[procnames]{listings}
\usepackage{color}
\usepackage{todonotes}
 
 

% See the ``Article customise'' template for come common customisations
\newcommand{\refeq}[1]{Equation~\ref{eq:#1}}
\newcommand{\reffig}[1]{Figure~\ref{fig:#1}}
\newcommand{\reftab}[1]{Table~\ref{tab:#1}}
\newcommand{\refsec}[1]{\textsection\ref{sec:#1}}
\newcommand{\newsec}[1]{\section{#1}\noindent}
%\newcommand{\newsec}[2]{\section{#1}\label{sec:#2}\noindent}
\newcommand{\newsubsec}[2]{\subsection{#1}\label{sec:#2}\noindent}
\newcommand{\argmax}{\operatornamewithlimits{argmax}} 
\newcommand{\argmin}{\operatornamewithlimits{argmin}} 

\makeatletter         
\def\@maketitle{   % custom maketitle 
\begin{center}%
{\bfseries \@title}%
{\bfseries \@author}%
\end{center}%
\smallskip \hrule \bigskip }

% custom section 
\renewcommand{\section}{\@startsection
{section}%                   % the name
{1}%                         % the level
{0mm}%                       % the indent
{-0.8\baselineskip}%            % the before skip
{0.3\baselineskip}%          % the after skip
{\bfseries\large}}% the style

% custom subsection 
\renewcommand{\subsection}{\@startsection
{subsection}%                   % the name
{2}%                         % the level
{0mm}%                       % the indent
{-0.8\baselineskip}%            % the before skip
{0.3\baselineskip}%          % the after skip
{\bfseries\large}}% the style

\renewcommand{\paragraph}{%
  \@startsection{paragraph}{4}%
  {\z@}{1.5ex \@plus 1ex \@minus .2ex}{-1em}%
  {\normalfont\normalsize\bfseries}%
}\makeatother

% taken from MdL
\newenvironment{titlemize}[1]{%
    \paragraph{#1}
    \begin{itemize}
        \setlength\itemsep{0pt}}
    {\end{itemize}}


%\title{{\LARGE Universal Parser (UP)}\\[-8mm]
%\includegraphics[height=8mm]{RUPA}~~~~~~~~~~~~~~~~~~~~~~~~~~~~~~~~~~~~~~~~~~~~~~~~~~~~~~~~~~~~~~~~~~~~~~~~~~~\includegraphics[height=8mm]{RUPA}}
\title{{\LARGE Natural Language Processing}\\[1.5mm]{\large Assignment 2: Advanced Text Processing}}%\\Lemmatisation and Part-of-Speech-Tagging}}
\author{}
\date{} % delete this line to display the current date

%%% BEGIN DOCUMENT
\begin{document}

\definecolor{keywords}{RGB}{255,0,90}
\definecolor{comments}{RGB}{0,0,113}
\definecolor{red}{RGB}{160,0,0}
\definecolor{green}{RGB}{0,150,0}
\definecolor{UUlight}{RGB}{230,230,230}
\definecolor{UUmedium}{RGB}{190,190,190}
\definecolor{UUdark}{RGB}{130,130,130}
\definecolor{UUred}{RGB}{153,0,0}

\lstset{language=Python, 
        basicstyle=\ttfamily\small, 
        keywordstyle=\color{keywords},
        commentstyle=\color{comments},
        stringstyle=\color{red},
        showstringspaces=false,
        identifierstyle=\color{green},
        procnamekeys={def,class}}

\maketitle
%\tableofcontents
%\vspace{3mm}

\section{Introduction}
\noindent This assignment involves material from lectures 5, 6 and
7. You should have watched the relevant videos, read the relevant
chapters in the textbook and made a serious attempt at completing the
relevant labs before you attempt this assignment. If you feel you have
done that and still find the instructions unclear, you are welcome to
email the course teachers and/or go to office hours to ask for help.
The assignment is split into 2 sections, one about lexical semantics,
one about semantic role labelling. Each section is worth 10 points. We
expect between half a page and a page for each section, except when
stated otherwise. Please do not submit more than 4 pages overall.  You
answers for each section should be self-contained.  \todo[inline,
color=blue!40]{FC: Right now, the assignment consists of many smaller
  questions. Can/Should we make them fewer bigger questions?}

\newsec{POS-Tagging}%
\begin{itemize}
\item In Lab 5, you have tuned a tagger. Based on the best version of
  your tagger, you should perform a manual error analysis where you go
  through at least 5 sentences and comment on the errors made by the
  tagger. Are the mistagged words genuinely ambiguous? Why do you
  think they were mistagged? Is it possible that some of the words are
  mistagged in the gold standard? \textcolor{UUred}{[ca 1/2 page]}.
\item What tagsets exist for your native language? List the ones you
  can find (spend not more than 15min on the search) and describe one
  of them. \textcolor{UUred}{[5-10
    sentences]} %\\\textbf{\textcolor{blue}{[tagsets $|$
                            %own web search]}}
\item Is it neccessary to tokenize text before tagging it? Please
  motivate your answer and give at least one
  example. \textcolor{UUred}{[5-10
    sentences]} %\\\textbf{\textcolor{blue}{[Read the book]}}
\item In the HMM lab we have investigated key sequences and produced
  words. What do these correspond to when using HMMs for POS-tagging?
  Please give a short motivation for your
  answer. \textcolor{UUred}{[5-10 sentences]}
  %\\\textbf{\textcolor{blue}{[HMMs $|$ Back-ref to lab]}}
\end{itemize}

\todo[inline, color=blue!40]{FC: We should discuss whether it is OK to
  keep these individual recommendations for answer lengths. I saw that
  I am the only one using it.}

\newsec{Lemmatisation}%
%What are they supposed to do?

\begin{itemize}
\item In Lab 7, you have tuned a lemmatizer. Based on the best version
  of your lemmatizer, you should do a manual analysis of remaining
  errors. Describe at least 5 error types and discuss how they could
  be tackled in a more sophisticated lemmatizer. \textcolor{UUred}{[ca
    1/2 page]}
\item Lemmatizers are often implemented as finite-state-transducers
  (FSTs). While this kind of implementation is beyond the scope of
  this course, Chapter 3.5 (FSTs for Morphological Parsing) of our
  course book gives examples of how FSTs can be visualised. Draw an
  FST based on the initial lemmatizer we gave you in Lab 7 (repeated
  on the last page of this assignment) that can analyse the following
  words: \texttt{cats NOUN}, \texttt{jumped VERB}, \texttt{higher
    ADJ}. You can either draw the FST 1) by hand, take a picture,
  transform it to \texttt{.pdf}, or 2) using a drawing program
  (e.g. xfig or MS Paint) and transform the output into
  \texttt{.pdf}. The drawing must be included in your submission, it
  should \textbf{not} be submitted as a separate file. Please describe
  your drawing. \textcolor{UUred}{[5-10 sentences]}
\item Why is it more difficult to tag morphologically rich languages?
  Please reflect and motivate your answer. \textcolor{UUred}{[min. 10
    sentences]}
\end{itemize}

\clearpage
\section{Grading Criteria}
To pass the assignment, you must meet all the basic criteria on all
subparts of the assignment.  To get VG, you must in addition meet some
of the additional criteria for most of subparts.

\begin{titlemize}{Basic Criteria}
    \item Answers are given in understandable English.
    \item Answers are stated clearly and coherently.
    \item Answers are essentially coherent.
\end{titlemize}
\begin{titlemize}{Additional Criteria}
    \item Answers are well motivated.
    \item Answers are well illustrated.
    \item Answers reveal extensive knowledge of the textbook chapter(s).
\end{titlemize}

\begin{titlemize}{Points}
    \item 1-4: poor on each of the basic criteria
    \item 5-6: fair on each of the basic criteria
    \item 7: at least fair on each of the basic criteria and fair on some of
        the additional criteria
    \item 8: at least good on each of the basic criteria and at least fair on
        each of the additional criteria
    \item 9: at least good on each of the basic criteria and at least fair and
        most times good or excellent on each of tem the additional criteria
    \item 10: good or excellent on each of the basic and additional criteria.
\end{titlemize}

\section{Submit the assignment}
\todo[inline, color=blue!40]{FC: we should decide whether or not we
  want them to submit in LaTeX}
\noindent
Submit your assignment as a pdf file named
firstname\_lastname\_assignment\_2.pdf. It should follow the style and
margins given in the example submission even if not created with
LaTeX. The submission is due on \textit{studentportalen} before
Wednesday 29th November 20h00. Later submissions will be considered
failed submissions and assessed after the final re-submission deadline
on January 15th. To pass the assignment, you must have answered both
questions, reached at least 5 points in each question and at least 12
overall. To get VG, you should obtain at least 8 in each question and
at least 18 points overall.  \todo[inline, color=blue!40]{FC: maybe
  say ``must have answered \underbar{all} questions?''. Sometimes, the
  large questions contain sub-questions.}  \todo[inline,
color=blue!40]{FC: I added the re-submission policy here and will add
  it to the webpage as well.}


  % \item Reflect on the importance of lemmatisation for your native
%  language and at least one foreign language you know.\\
%  \textbf{\textcolor{blue}{$\rightarrow$ [productivity $|$
%      morphological richness]}}
%\item How difficult do you estimate lemmatisation to be for your
%  native language (and why)? \\\textbf{\textcolor{blue}{$\rightarrow$
%      [ambiguity]}}
%\item What is an FST? What is it used for and why is it useful? (ca. 1/2 page)\\
%  \textbf{\textcolor{blue}{$\rightarrow$ [Technical background $|$
%      Read the book, esp. 3.4-3.7]}}
%\newsec{Illustrate an FST} %

%\newsec{Error analysis}{error}%

% \newsec{Summing up}%

% \begin{itemize}
% \item Why is it more difficult to tag morphologically rich
%   languages?\\\textbf{\textcolor{blue}{[Morphological Productivity
%       $\rightarrow$ Unknown words in tag lexicon]}}
% \item What are the possible advantages/disadvantages of first
%   performing lemmatisation and then POS-tagging and doing it vice
%   versa? \\\textbf{\textcolor{blue}{$\rightarrow$ [error propagation
%       $|$ reduce/enhance ambiguity $|$ out-of-the-box thinking]}}
% \end{itemize}

\begin{figure}
\begin{center}
  \fbox{
\lstinputlisting{code/lemmatizer1.py}
}
\caption{Skeleton code for a rule-based lemmatizer, taken from Lab 7.}
\end{center}
\end{figure}


\end{document}