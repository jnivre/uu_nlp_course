
\documentclass[11pt]{article}
%\usepackage[top=20mm,left=20mm,right=20mm,bottom=15mm,a4paper]{geometry} % see geometry.pdf on how to lay out the page. There's lots.
\usepackage[top=20mm,left=20mm,right=20mm,bottom=15mm,headsep=15pt,footskip=15pt,a4paper]{geometry} % see geometry.pdf on how to lay out the page. There's lots.
%\geometry{a4paper} % or letter or a5paper or ... etc
% \geometry{landscape} % rotated page geometry
\usepackage[round]{natbib}
\setlength{\bibsep}{0.0pt}
\usepackage{color}
\usepackage{times}
%\usepackage[T1]{fontenc}
%\usepackage{mathptmx}
\usepackage{tikz-dependency}
\usepackage{enumitem}
%\usepackage{times}

\usepackage[procnames]{listings}
\usepackage{color}
\usepackage{todonotes}

\newenvironment{titlemize}[1]{%
    \paragraph{#1}
    \begin{itemize}
        \setlength\itemsep{0pt}}
    {\end{itemize}}
 
 

% See the ``Article customise'' template for come common customisations
\newcommand{\refeq}[1]{Equation~\ref{eq:#1}}
\newcommand{\reffig}[1]{Figure~\ref{fig:#1}}
\newcommand{\reftab}[1]{Table~\ref{tab:#1}}
\newcommand{\refsec}[1]{\textsection\ref{sec:#1}}
\newcommand{\newsec}[2]{\section{#1}\label{sec:#2}\noindent}
\newcommand{\newsubsec}[2]{\subsection{#1}\label{sec:#2}\noindent}
\newcommand{\argmax}{\operatornamewithlimits{argmax}} 
\newcommand{\argmin}{\operatornamewithlimits{argmin}} 

\makeatletter         
\def\@maketitle{   % custom maketitle 
\begin{center}%
{\bfseries \@title}%
{\bfseries \@author}%
\end{center}%
\smallskip \hrule \bigskip }

% custom section 
\renewcommand{\section}{\@startsection
{section}%                   % the name
{1}%                         % the level
{0mm}%                       % the indent
{-0.8\baselineskip}%            % the before skip
{0.3\baselineskip}%          % the after skip
{\bfseries\large}}% the style

% custom subsection 
\renewcommand{\subsection}{\@startsection
{subsection}%                   % the name
{2}%                         % the level
{0mm}%                       % the indent
{-0.8\baselineskip}%            % the before skip
{0.3\baselineskip}%          % the after skip
{\bfseries\large}}% the style

\renewcommand{\paragraph}{%
  \@startsection{paragraph}{4}%
  {\z@}{1.5ex \@plus 1ex \@minus .2ex}{-1em}%
  {\normalfont\normalsize\bfseries}%
}\makeatother

%\title{{\LARGE Universal Parser (UP)}\\[-8mm]
%\includegraphics[height=8mm]{RUPA}~~~~~~~~~~~~~~~~~~~~~~~~~~~~~~~~~~~~~~~~~~~~~~~~~~~~~~~~~~~~~~~~~~~~~~~~~~~\includegraphics[height=8mm]{RUPA}}
\title{{\LARGE Natural Language Processing}\\[1.5mm]{\large Assignment 1: Words}}
\author{}
\date{} % delete this line to display the current date

%%% BEGIN DOCUMENT
\begin{document}

\definecolor{keywords}{RGB}{255,0,90}
\definecolor{comments}{RGB}{0,0,113}
\definecolor{red}{RGB}{160,0,0}
\definecolor{green}{RGB}{0,150,0}

\lstset{language=Python, 
        basicstyle=\ttfamily\small, 
        keywordstyle=\color{keywords},
        commentstyle=\color{comments},
        stringstyle=\color{red},
        showstringspaces=false,
        identifierstyle=\color{green},
        procnamekeys={def,class}}

\maketitle
%\tableofcontents
%\vspace{3mm}
\section{Introduction}
\indent This assignment involves material from lectures 1 to 4. You
should have watched the relevant videos, read the relevant chapters in
the textbook and made a serious attempt at completing the relevant
labs before you attempt this assignment.  If you feel you have done
that and still find the instructions unclear, you are welcome to email
the course teachers and/or go to office hours to ask for help.  The
assignment is split into 2 sections, one about tokenization, one about
language modelling and probability and statistics. Each section is
worth 10 points. We expect between half a page and a page for each
section. Please do not submit more than 4 pages overall.  You answers
for each section should be self-contained.
\section{Tokenization}
In the videos and the book chapters, you learned about regular expressions (if
you did not already know all about them before that ;)). In the labs, you got
to play with regular expressions in the context of tokenization.
You started with a very simple tokenizer and gradually refined it by looking at
the errors it was making. 
If you have not done so already, try to improve it so that it reaches at least
95\% precision and recall.  
Describe your approach to improving it. You may illustrate the problems you
dealt with by adding snippets of code.

\todo{Add info on how they will receive the new test set}
You will receive a new test set {\tt dev2-raw.txt} to run your tokenizer on.  
Write down for yourself your expectation of how good you think your tokenizer
is going to be before running it.  
Compare the output of your system against {\tt dev2-gold-sent.txt} and discuss
the result. Are there still cases of under/oversplitting? If so, can you
identify the reason? Can you identify difficult challenges?

\section{Language Modelling}
\indent In Lab 3, you learned about how to use MLE for language modelling.
Hopefully, in the lab, you saw its limitations. 
In Lab 4, you experimented with smoothing methods.\\
\indent Define MLE and discuss its limitations. Explain the principles behind
smoothing and how they help circumvent the limitations of MLE. Illustrate this
with examples. (You may take them from the Sherlock Holmes text but use
different examples than those we asked you to look at in the labs.)\\
\indent Describe and compare different smoothing methods.  Illustrate your
answer with results from experiments you made in the lab. You may use the table
you produced during the lab.

\section{Grading Criteria}
\begin{titlemize}{Basic Criteria}
\item Answers are given in understandable English.
\item Answers are stated clearly and coherently.
\item Answers are essentially correct.
\end{titlemize}
\begin{titlemize}{Additional Criteria}
\item Answers are well motivated.
\item Answers are well illustrated.
\item Answers reveal extensive knowledge of the textbook chapter(s).
\end{titlemize}
To pass the assignment, you must meet all the basic criteria on all subparts of the assignment. 
To get VG, you must in addition meet some of the additional criteria for most of subparts.


\section{Submit the assignment}
%\todo[inline, color=blue!40]{FC: we should decide whether or not we
%  want them to submit in LaTeX}
\noindent
Submit your assignment as a pdf file named
firstname\_lastname\_assignment\_1.pdf. It should follow the style and
margins given in the example submission even if not created with
LaTeX. The submission is due on \textit{studentportalen} before Monday
November 29th at 20h00. Later submissions will be considered failed
submissions and assessed after the final re-submission deadline on
January 15th.\todo[inline, color=green!50]{FC: added submission format
  requirements}
% To pass the assignment, you
% must have answered both sections, reached at least 5 points in each
% section and at least 12 overall. To get VG, you should obtain at
% least 8 in each section and at least 18 points overall.
% \todo[inline, color=yellow!40]{FC: This sentence is a bit
%   redundant. They cannot get 18 points without getting at least 8
%   points in each question anyway, right? Moreover, I changed the
%   wording here from 'questions' to 'sections' in order to make it fit
%   with the text in the introduction where we say that each
%   \textbf{section} is worth 10 points.}\todo[inline,
% color=blue!40]{FC: maybe say ``must have answered \underbar{all}
%   questions?''. Sometimes, the large questions contain sub-questions.}
% \todo[inline, color=blue!40]{FC: I added the re-submission policy here
%   and will add it to the webpage as well.}

\end{document}
