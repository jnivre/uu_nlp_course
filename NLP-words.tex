
\documentclass[11pt]{article}
%\usepackage[top=20mm,left=20mm,right=20mm,bottom=15mm,a4paper]{geometry} % see geometry.pdf on how to lay out the page. There's lots.
\usepackage[top=20mm,left=20mm,right=20mm,bottom=15mm,headsep=15pt,footskip=15pt,a4paper]{geometry} % see geometry.pdf on how to lay out the page. There's lots.
%\geometry{a4paper} % or letter or a5paper or ... etc
% \geometry{landscape} % rotated page geometry
\usepackage[round]{natbib}
\setlength{\bibsep}{0.0pt}
\usepackage{color}
\usepackage{times}
%\usepackage[T1]{fontenc}
%\usepackage{mathptmx}
\usepackage{tikz-dependency}
\usepackage{enumitem}
%\usepackage{times}

\usepackage[procnames]{listings}
\usepackage{color}
\usepackage{todonotes}

\newenvironment{titlemize}[1]{%
    \paragraph{#1}
    \begin{itemize}}
            {\end{itemize}}
 
 

% See the ``Article customise'' template for come common customisations
\newcommand{\refeq}[1]{Equation~\ref{eq:#1}}
\newcommand{\reffig}[1]{Figure~\ref{fig:#1}}
\newcommand{\reftab}[1]{Table~\ref{tab:#1}}
\newcommand{\refsec}[1]{\textsection\ref{sec:#1}}
\newcommand{\newsec}[2]{\section{#1}\label{sec:#2}\noindent}
\newcommand{\newsubsec}[2]{\subsection{#1}\label{sec:#2}\noindent}
\newcommand{\argmax}{\operatornamewithlimits{argmax}} 
\newcommand{\argmin}{\operatornamewithlimits{argmin}} 

\makeatletter         
\def\@maketitle{   % custom maketitle 
\begin{center}%
{\bfseries \@title}%
{\bfseries \@author}%
\end{center}%
\smallskip \hrule \bigskip }

% custom section 
\renewcommand{\section}{\@startsection
{section}%                   % the name
{1}%                         % the level
{0mm}%                       % the indent
{-0.8\baselineskip}%            % the before skip
{0.3\baselineskip}%          % the after skip
{\bfseries\large}}% the style

% custom subsection 
\renewcommand{\subsection}{\@startsection
{subsection}%                   % the name
{2}%                         % the level
{0mm}%                       % the indent
{-0.8\baselineskip}%            % the before skip
{0.3\baselineskip}%          % the after skip
{\bfseries\large}}% the style

\renewcommand{\paragraph}{%
  \@startsection{paragraph}{4}%
  {\z@}{1.5ex \@plus 1ex \@minus .2ex}{-1em}%
  {\normalfont\normalsize\bfseries}%
}\makeatother

%\title{{\LARGE Universal Parser (UP)}\\[-8mm]
%\includegraphics[height=8mm]{RUPA}~~~~~~~~~~~~~~~~~~~~~~~~~~~~~~~~~~~~~~~~~~~~~~~~~~~~~~~~~~~~~~~~~~~~~~~~~~~\includegraphics[height=8mm]{RUPA}}
\title{{\LARGE Natural Language Processing}\\[1.5mm]{\large Assignment 1: Words}}
\author{}
\date{} % delete this line to display the current date

%%% BEGIN DOCUMENT
\begin{document}

\definecolor{keywords}{RGB}{255,0,90}
\definecolor{comments}{RGB}{0,0,113}
\definecolor{red}{RGB}{160,0,0}
\definecolor{green}{RGB}{0,150,0}
 
\lstset{language=Python, 
        basicstyle=\ttfamily\small, 
        keywordstyle=\color{keywords},
        commentstyle=\color{comments},
        stringstyle=\color{red},
        showstringspaces=false,
        identifierstyle=\color{green},
        procnamekeys={def,class}}

\maketitle
%\tableofcontents
%\vspace{3mm}
\section{Introduction}
\indent This assignment involves material from lectures 1 to 4. We recommend
that you have watched the relevant videos, read the relevant chapters in the
textbook and made a serious attempt at completing the relevant labs before you
attempt this assignment.
\todo[inline]{What do you think about adding: if you feel you have done that
and still find the instructions unclear, you are welcome to email the course
teachers and/or attend office hours}
The assignment is split into 2 sections, one about tokenization, one about
language modelling and probability and statistics. Each section is worth 10
points. We expect between half a page and a page for each section. Please do
not submit more than 4 pages overall. \todo{adjust the question sizes?}
You answers for each section should be self-contained.
\section{Tokenization}
In the videos and the book chapters, you learned about regular expressions (if
you did not already know all about them before that ;)). In the labs, you got
to play with regular expressions in the context of tokenization.
You started with a very simple tokenizer and gradually refined it by looking at
the errors it was making. Analysing errors is an important part of the job of
NLP practitioners. In this part of the assignment, we ask you to analyse the
errors of your tokenizer.

You will receive a new test set {\tt dev2-raw.txt} to run your tokenizer on.  
If you are not yet satisfied with your tokenizer, you may want to improve it on
the test set of the lab.
\todo[inline]{Should they aim for a score? In last year's set-up they were
encouraged to reach high accuracy. I think it's a good encouragement to have
but I'm not sure how to implement this in the assignment.}
Compare the output of your system against {\tt dev2-gold-sent.txt} and discuss
the result. Are there still cases of under/oversplitting? If so, can you
identify the reason? Can you identify difficult challenges?
\todo{Should we add this on kasus at the time of handing in the assignment?}

\section{Language Modelling}
\indent In Lab 3, you learned about how to use MLE for language modelling.
Hopefully, in the lab, you saw its limitations. 
In Lab 4, you experimented with smoothing methods.\\
\indent Define MLE and discuss its limitations. Explain the principles behind
smoothing and how they help circumvent the limitations of MLE. Illustrate this
with examples. (You may take them from the Sherlock Holmes text but use
different examples than those we asked you to look at in the labs.)\\
\indent Discuss different smoothing methods and their advantages and
disadvantages. Use your results table from the lab to illustrate your points.

\section{Grading Criteria}
\begin{titlemize}{Basic Criteria}
    \item The problem is clearly stated and the student demonstrates an
        understanding of the problem.
    \item The answer is coherent
    \item The quality of the answer is in understandable English
\end{titlemize}
\begin{titlemize}{Additional Criteria}
    \item The problem is well illustrated
    \item Solutions to the problem are described
    \item The student demonstrates extensive knowledge of the textbook chapter
\end{titlemize}

\begin{titlemize}{Points}
    \item 1-4: poor on each of the basic criteria
    \item 5-6: fair on each of the basic criteria
    \item 7: at least fair on each of the basic criteria and fair on some of
        the additional criteria
    \item 8: at least good on each of the basic criteria and at least fair on
        each of the additional criteria
    \item 9: at least good on each of the basic criteria and at least fair and
        most times good or excellent on each of tem the additional criteria
    \item 10: good or excellent on each of the basic and additional criteria.
\end{titlemize}

\section{Submit the assignment}
\todo{ADD ME}
To pass the assignment, you must have answered both questions, reached at least
5 points in each question and at least 12 overall.
To get VG, you should obtain at least 8 in each question and at least 18 points
overall.

\end{document}
